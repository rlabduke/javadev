%% LyX 1.6.1 created this file.  For more info, see http://www.lyx.org/.
%% Do not edit unless you really know what you are doing.
\documentclass[english]{report}
\usepackage{mathptmx}
\usepackage{helvet}
\usepackage{courier}
\usepackage[T1]{fontenc}
\usepackage[latin9]{inputenc}
\setcounter{secnumdepth}{3}
\setcounter{tocdepth}{3}


%%%%%%%%%%%%%%%%%%%%%%%%%%%%%% LyX specific LaTeX commands.
%% Because html converters don't know tabularnewline
\providecommand{\tabularnewline}{\\}

\usepackage{babel}

\begin{document}
\vfill{}



\title{The Chiropraxis tools manual}

\maketitle
\vfill{}



\author{Ian W. Davis}


\chapter{The Tools}


\section{BACKRUB Tool: Protein chiropraxis\label{backrub-tool}}

The BACKRUB tool is used to adjust short segments of protein backbone
without disturbing the surrounding structure. It allows you select
one residue, and rotate it around the axis between its neighboring
C-alphas. Using BACKRUB requires that \texttt{chiropraxis.jar} be
present in the \texttt{plugins/} folder.

Create a backrub by control-clicking the alpha carbon of the central
residue. To use the dial control, click and drag the pointer to the
desired value. For finer control you can shift-drag along an imaginary
slider (this behavior is somewhat like the virtual dials in O). To
reset the dial to its starting value, double click the dial face.
The dial can also be adjusted from the graphics window using the mouse
wheel or arrow keys; see below.

The BACKRUB panel provides feedback on the geometric quality of the
current model, displaying residues names, deviation of the tau angle
(N-CA-C) from ideality, and the position of the residue on the Ramachandran
plot (phi, psi, and favored/allowed/outlier). Deviant geometry is
highlighted in red.

\emph{BACKRUB is described in the following publication: I.W. Davis,
W.B. Arendall III, D.C. Richardson, and J.S. Richardson (2006) \textquotedbl{}The
backrub motion: How protein backbone shrugs when a sidechain dances\textquotedbl{}
Structure 14, 265-274.}


\subsection{Command reference}

\noindent \begin{center}
\begin{tabular}{|c|c|}
\hline 
\multicolumn{2}{|c|}{Mouse clicks}\tabularnewline
\hline
\hline 
Normal & Mark and identify point; make measurement (pick)\tabularnewline
\hline 
With Shift & Center on selected point (pickcenter)\tabularnewline
\hline 
With Ctrl & Select residue for BACKRUB (on a C-alpha); clear current selection
(elsewhere)\tabularnewline
\hline 
With Shift+Ctrl & -\tabularnewline
\hline
\end{tabular}
\par\end{center}

\noindent \begin{center}
\begin{tabular}{|c|c|}
\hline 
\multicolumn{2}{|c|}{Mouse drags}\tabularnewline
\hline
\hline 
Normal & Rotate around X and Y axes; Z-rotate (pinwheel) near top of screen\tabularnewline
\hline 
With Shift & Adjust zoom (up/down); Adjust clipping (left/right) \tabularnewline
\hline 
With Ctrl & Translate in X-Y plane (flatland); Z-translate near top of screen\tabularnewline
\hline 
With Shift+Ctrl & Rotate around Y axis only\tabularnewline
\hline
\end{tabular}
\par\end{center}

\noindent \begin{center}
\begin{tabular}{|c|c|}
\hline 
\multicolumn{2}{|c|}{Mouse wheel / Up \& Down arrow keys}\tabularnewline
\hline
\hline 
Normal & Adjust dial (angle of rotation)\tabularnewline
\hline 
With Shift & Adjust clipping\tabularnewline
\hline 
With Ctrl & Adjust zoom\tabularnewline
\hline 
With Shift+Ctrl & -\tabularnewline
\hline
\end{tabular}
\par\end{center}


\section{C-alpha Hinges\label{hinges-tool}}

Hinges is the generalized version of BACKRUB. It allows you to select
any continuous region of backbone that joins two alpha carbons, and
then rotate that segment of backbone around an axis drawn between
those C-alphas. Individual peptides can also be rotated. Create a
hinge by control-clicking the two alpha carbons that act as its endpoints
or anchors.

Regions longer than two peptides (\emph{i.e.}, what BACKRUB handles)
are seldom useful in fitting structures. Thus, this tool has been
largely superceded by BACKRUB.


\section{Shear Tool: Helical winding plus unwinding \label{shear-tool}}

The shear tool can be used to modify backbone, most commonly helix,
in a manner almost completely orthogonal to the backrub. A shear shifts
three peptides instead of two like a backrub, and moves the central
peptide nearly parallel to the chain direction rather than perpendicular
to it. Use Ctrl+Click (a.k.a. middle-click) to select a residue; it,
its preceding residue, and its two following residues will become
the active {}``molten'' state. Next, drag the primary shear dial
for the main motion, then optionally drag the three peptide rotation
dials to alleviate some strain. Functionality is generally similar
to the backrub tool.

\emph{Authored by Daniel Keedy} \emph{(daniel.keedy AT duke.edu)}


\section{Tweak Phi/Psi: Visualizing possible backbone changes \label{phisi-tool}}

The tweak phi/psi tool is useful for slightly altering the backbone
dihedrals phi and psi. Use Ctrl+Click (a.k.a. middle-click) to select
a residue and its adjacent region as the active {}``molten'' state.
The length of the active region can be customized by entering a number
of residues in the box and pressing enter. Next, drag the phi/psi
dials to rotate the molten region around either the phi or psi axis.
Note that this tool makes no attempt to close the resulting chain
break or to alleviate the steric clashes it usually introduces, and
thus should not be employed as a fitting tool. However, it can be
useful for visualizing the local effects of small changes in phi/psi
and simply building intuition.

\emph{Authored by Daniel Keedy} \emph{(daniel.keedy AT duke.edu)}


\section{Sidechain Rotator: Refitting protein models\label{scrot-tool}}

The Sidechain Rotation tool is useful for doing interactive refitting
of protein models in conjunction with the Hinges tool and the model
manager. Use Ctrl+Click (a.k.a. middle-click) to select a sidechain
to rotate, then pick from a list of predefined rotamers or set the
angles by hand using the dials. The rotamericity of the current sidechain
is monitored in the bottom right corner; it indicates how frequently
(if ever!) the current conformation is found in well-determined structures.

To use the dial control, click and drag the pointer to the desired
value. For finer control you can shift-drag along an imaginary slider
(this behavior is somewhat like the virtual dials in O). To reset
the dial to its starting value, double click the dial face.

Sidechain Rotator and the Hinges tool play well together: you can
use both on the same residue(s) at the same time. It's often easiest
to establish which sidechains will be rotatable before beginning to
move the backbone, but it's possible to establish the moving parts
in any order.


\section{Sidechain Mutator: Redesigning protein models\label{scmut-tool}}

The Sidechain Mutation tool is useful for doing interactive refitting
of protein models in conjunction with the Hinges and Rotator tools
and the model manager. Use Ctrl+Click (a.k.a. middle-click) to select
a sidechain to mutate, then pick from a list of known amino acid types.
If you select histidine, you will also be prompted to choose a protonation
state.

Sidechain Mutator plays well with the other tools, but you cannot
mutate the model while any part of it is mobile ({}``molten'').
Thus, it's often easiest to make mutations first, and then refit the
new sequence.


\section{Model Manager: Editing a macromolecular model\label{modelman-plugin}}

This facility is for doing molecular modeling based on some starting
model, usually a PDB file. The Model Manager handles opening and saving
these files, and is required for the operation of the following tools:
\begin{itemize}
\item Backrub
\item C-alpha hinges
\item Shear
\item Tweak phi/psi
\item Sidechain rotator
\item Sidechain mutator
\end{itemize}
The model manager also tracks the changes made to the model and allows
near-unlimited undo. In addition, it provides access to dynamic visualizations,
like Probe dots and NOE constraints. Using this feature requires that
your OS can find Probe -- \emph{i.e.}, that either Probe resides in
the same directory as \texttt{king.jar} and is named either \texttt{probe}
or \texttt{probe.exe}; or that Probe is somewhere on your PATH and
is named \texttt{probe} or \texttt{probe.exe}. The same is true of
\texttt{noe-display}.

There are several special symbols that can be inserted into the Probe
and NOE command lines. Their meanings are as follows:
\begin{description}
\item [{\{pdbfile\}}] The fully-qualified name of the {}``base'' PDB
file; \emph{i.e.} the one with all changes except the currently molten
ones. (Those are piped in via standard input.)
\item [{\{molten\}}] The list of molten residues, separated by commas:
1, 2, 3, ...
\item [{\{center\}}] The coordinates of the current center of view, in
real space: x, y, z. This can be used for the \texttt{within} \emph{distance}
\texttt{of} \emph{x, y, z} selection statement. Note that visualizations
aren't automatically updated when the center of view changes.
\end{description}

\chapter{Copyright \& acknowledgments}


\section{Copyright}

The Chiropraxis code and all its associated original resources and
documentation are copyright (C) 2002-2010 by Ian W. Davis, Daniel
A. Keedy, and Vincent B. Chen.


\section{Revision status}

This manual was last updated 1 September 2010 by DAK for Chiropraxis
version 1.07.
\end{document}
